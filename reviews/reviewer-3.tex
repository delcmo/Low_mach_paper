\documentclass{article}
%%%%%%%%%%%%%%%%%%%%%%%%%%%%%%%%%%%%%%%%%%%%%%%%%%%%%%%%%%%%%%%%%%%%%%%%%%%%%%%%%%%%%%%%%%%%%%%%%%%%%%%%%%%%%%%%%%%%%%%%%%%%%%%%%%%%%%%%%%%%%%%%%%%%%%%%%%%%%%%%%%%%%%%%%%%%%%%%%%%%%%%%%%%%%%%%%%%%%%%%%%%%%%%%%%%%%%%%%%%%%%%%%%%%%%%%%%%%%%%%%%%%%%%%%%%%
\usepackage{amsmath,amssymb}
% more math
\usepackage{amsfonts}
\usepackage{amssymb}
\usepackage{amstext}
\usepackage{amsbsy}
%\usepackage{dtklogos}

\usepackage{color}
\newcommand{\mt}[1]{\marginpar{\small #1}}
%%%%%%%%%%%%%%%%%%%%%%%%%%%%%%%%%%%%%%%%%%%%%%%%%%%%%%%%%%%%%%%%%%%%
% new commands
\newcommand{\nc}{\newcommand}
% operators
\renewcommand{\div}{\vec{\nabla}\! \cdot \!}
\newcommand{\grad}{\vec{\nabla}}
% latex shortcuts
\newcommand{\bea}{\begin{eqnarray}}
\newcommand{\eea}{\end{eqnarray}}
\newcommand{\be}{\begin{equation}}
\newcommand{\ee}{\end{equation}}
\newcommand{\bal}{\begin{align}}
\newcommand{\eali}{\end{align}}
\newcommand{\bi}{\begin{itemize}}
\newcommand{\ei}{\end{itemize}}
\newcommand{\ben}{\begin{enumerate}}
\newcommand{\een}{\end{enumerate}}
% DGFEM commands
\newcommand{\jmp}[1]{[\![#1]\!]}                     % jump
\newcommand{\mvl}[1]{\{\!\!\{#1\}\!\!\}}             % mean value
\newcommand{\keff}{\ensuremath{k_{\textit{eff}}}\xspace}
% shortcut for domain notation
\newcommand{\D}{\mathcal{D}}
% vector shortcuts
\newcommand{\vo}{\vec{\Omega}}
\newcommand{\vr}{\vec{r}}
\newcommand{\vn}{\vec{n}}
\newcommand{\vnk}{\vec{\mathbf{n}}}
\newcommand{\vj}{\vec{J}}
% extra space
\newcommand{\qq}{\quad\quad}
% common reference commands
\newcommand{\eqt}[1]{Eq.~(\ref{#1})}                     % equation
\newcommand{\fig}[1]{Fig.~\ref{#1}}                      % figure
\newcommand{\tbl}[1]{Table~\ref{#1}}                     % table

\newcommand{\ud}{\,\mathrm{d}}

\newcommand{\tcr}[1]{\textcolor{red}{#1}}
\newcommand{\tcb}[1]{\textcolor{blue}{#1}}

\def\BibTeX{{\rm B\kern-.05em{\sc i\kern-.025em b}\kern-.08em
    T\kern-.1667em\lower.7ex\hbox{E}\kern-.125emX}}
%%%%%%%%%%%%%%%%%%%%%%%%%%%%%%%%%%%%%%%%%%%%%%%%%%%%%%%%%%%%%%%%%%%%

\begin{document}
\bibliographystyle{elsarticle-num}
\begin{center}
{ \Large Answers to Reviewer \#3}
\end{center}

\bigskip

\noindent Ms. Ref. No.: CAF-D-14-00634\\
Title: ``Entropy-based viscous regularization for the multi-dimensional Euler equations in low-Mach and transonic flows', \\
{\it Computers and Fluids}\\

\bigskip
\bigskip

{
\color{blue}
This looks like it a revised manuscript modified on the basis of one
extensive review.
This work is largely derivative of the work of Geurmond and Popov and
produces a reasonable extension of their work. The revision is largely
successful per the comments of that reviewer. A few issue remain
unaddressed and I will highlight these below.
}

\bigskip

{
\color{blue}
1. In the introduction the description of the stabilization methods
available for hyperbolic PDEs is a bit pedestrian. There are dissipation
approaches typified by artificial viscosity and Riemann solvers, and
"limiters" associated with high-order discretizations. Both are important,
but they are quite different. Two major themes exist (Van 2006 covers
this topic well, as well as Zalesak 1997).
Van Leer, B. (2006). Upwind and high-resolution methods for
compressible flow: From donor cell to residual-distribution schemes.
Communications in Computational Physics, 1(192-206), 12.
Zalesak, Steven T. "Introduction to "Flux-corrected transport. I. SHASTA,
a fluid transport algorithm that works"." Journal of Computational Physics
135.2 (1997): 170-171.}

We rewrote the second paragraph of the introduction dealing with the stabilization techniques
in order to discuss and include additional references in lines 20 and 21. Thank you for pointing this out.
%\tcr{have you read Bram van Leer 2006 and added something related to this?} \tcb{I added the paper by Bram van Leer}

\bigskip


{
\color{blue}
2. The connections to the work of Cook and Cabot seem to be very
strong. These should be mentioned and examined.
Cook, Andrew W., and William H. Cabot. "Hyper-viscosity for shock turbulence
interactions." Journal of Computational Physics 203.2 (2005):
379-385.
Cook, Andrew W., and William H. Cabot. "A high-wave number viscosity
for high-resolution numerical methods." Journal of Computational Physics
195.2 (2004): 594-601.
Fiorina, Benoit, and Sanjiva K. Lele. "An artificial nonlinear diffusivity
method for supersonic reacting flows with shocks." Journal of
Computational Physics 222.1 (2007): 246-264.}

Thank you for these references. In these articles, an artificial viscosity is added to the momentum
and energy equations only, by means of grid-dependent viscosity coefficients. These techniques share some common thread with
the entropy viscosity technique and we have added some references to them in the introduction between lines 26 and 28.
However, as Guermond \& Popov have shown in our References \cite{jlg1, jlg2}, viscosity should also be added to the continuity equation and, in that sense, the entropy-viscosity method differ from the others.\\
We are also aware of some of ongoing work to employ the entropy-viscosity technique for turbulence modeling \tcr{Travis Thomson dissertation} and added such a reference \tcr{Do it}. \tcb{not done}
 %understanding of the work of Cook and Cabot is that it was developed for both shock and turbulence capturing. The entropy viscosity method seems to be a good candidate for turbulence capturing as well . However, since this paper does not deal with turbulence capturing but only with shock capturing, we feel that the reader might get confused by adding the above references.
\bigskip


{
\color{blue}
3. I'm surprised that Lax's work on vanishing viscosity is never mentioned
nor the numerical impact of it.
Lax, Peter D. Hyperbolic systems of conservation laws and the
mathematical theory of shock waves. Vol. 11. SIAM, 1973.
Harten, Amiram, et al. "On finite difference approximations and entropy
conditions for shocks." Communications on pure and applied mathematics
29.3 (1976): 297-322. 
}

We added the two above references to the introduction in line 17. We had initially referenced Leveque's work. Thank you for pointing this out.
\bigskip


{
\color{blue}
4. The issue of maintaining high-order accuracy does not matter once
shocks are formed. (second-order barely meets this condition!). The
accuracy drops to 1st order with shocks, and lower if the wave is linearly
discontinuous. I find the reference here to be odd and suggest using
something more classical in the shock community.
Majda, Andrew, and Stanley Osher. "Propagation of error into regions of
smoothness for accurate difference approximations to hyperbolic
equations." Communications on Pure and Applied Mathematics 30.6
(1977): 671-705.
Banks, Jeffrey W., T. Aslam, and W. J. Rider. "On sub-linear convergence
for linearly degenerate waves in capturing schemes." Journal of
Computational Physics 227.14 (2008): 6985-7002.
}

Thank you for pointing out the above references. We have added these references to the paper in the numerical results section in line 359. \tcr{will need to highlight this in the final PDF we re-submit}
\bigskip


{
\color{blue}
5. The bounding viscosity is actually larger than upwind, it is Rusanov or
Local Lax-Friedrichs viscosity determined by the largest wave speed in the
system. This could become very large and potentially damaging as the
flow becomes low-Mach. Is this problem encountered? Or is it mitigated?
This would seem to be a major potential problem with the formulation.
Most of the time the dissipation proportional to the material velocity
would seem to be the relevant one for the dynamics.
}

We have adopted this terminology for the bounding viscosity. This is more natural and
will help the reader. \\
You are correct that only using this bounding viscosity for low-Mach flows is inappropriate.  
The bounding viscosity is expected to kick in only in shocks; elsewhere the entropy viscosity is always orders of magnitude lower. Since this is only an upper bound, there are no issues in low-Mach flows. We have added a sentence to emphasize this in line 493.
\bigskip


{
\color{blue}
6. It would be good to see some modification and improvement of the
method in this paper. I only see an extension to low-Mach flows.
}

The method was improved in two significant ways: 
\begin{itemize}
\item the definition of the viscosity coefficients \underline{no longer} requires a functional form of the entropy function and thus, can be used with a wide range of equation of states and table look-ups. 
\item More importantly, a new entropy residual has been derived. It only depends on flow variables and enables to treatment of low-Mach flows with the entropy-viscosity formalism. This was not possible before.
\end{itemize}
\bigskip


{
\color{blue}
7. I find the quality of the line plots to be quite poor and not up to the
standards of publication. These should be improved.
}

We improved the quality of the 2-D plots that are presented in the numerical results section. 
\bigskip


{
\color{blue}
8. For the final few examples a more quantitative assessment of the
numerical accuracy and order of convergence of the method would be
quite welcome.
}
The compression corner yields a steady-state solution whose analytical pre- and post-shocks values are known. 
We compared our numerical simulation values against these values and show excellent agreement. 
We did not perform any convergence order as an analytical solution is not available \tcr{not true. also check at what angle a Mach stem begins}. 
\tcr{I do not want to use a fine mesh numerical solution as reference because I cannot be assured it will be a good
enough reference to do a convergence study. what do you think?\\} \tcb{we can probably run the code with a very fine mesh but it will take time and we are not sure that it will be fine enough to do the convergence test.\\}
Likewise, from potential flow theory, we know the exact value of the velocity at the top of the cylinder for the flow around a 
cylinder problem; we compared our numerical values with the exact ones and they are in excellent agreement.
With the meshes used, we have obtained an pressure variation that is exactly proportional to Mach$^2$, as expected from the theory.
%We could have performed a convergence test by using a numerical solution from a fine mesh as a reference and see if we recover the expected order of convergence, as it is done for some of the 1-d tests in the paper. \\

The test of a flow over a hump is the only test we do not have an exact solution for \tcr{is there one?}. \tcb{you can only compare against the numerical solution from an incompressible code (in the low-Mach limit).}
This test is often used in the literature to assess the symmetry of the steady-state solution when plotting the iso-Mach line 
for low-Mach flows. This is also commonly used for other test such as the NACA wing in \cite{LowMach1}.
\bigskip

\bibliography{mybibfile}

\end{document}

